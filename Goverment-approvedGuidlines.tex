\RequirePackage{luatex85}
\documentclass[paper=a4,fontsize=10.5pt]{jlreq}
\usepackage{amsmath,amsfonts,amssymb,mathtools,ascmac,bm,fancybox,calc,multicol,physics,array}
\usepackage[top=20truemm,bottom=20truemm,left=15truemm,right=15truemm]{geometry}
\usepackage{graphicx,color}
\usepackage{tikz,listing,wrapfig,float,xcolor}
\usepackage{url,subcaption,multirow,framed}
\usepackage[unicode,hidelinks,pdfusetitle]{hyperref}
\usepackage{luatexja-fontspec,lltjext}
\hypersetup{
    colorlinks=true,
    citecolor=black,
    linkcolor=black,
    urlcolor=blue
}

    \usetikzlibrary{intersections,calc,arrows.meta,backgrounds,shapes.geometric,shapes.misc,positioning,fit,graphs,arrows}
    \setlength{\columnsep}{5mm}

    \columnseprule=0.1mm
    \ltjsetparameter{jacharrange={-2}} %日本語以外を欧文扱い

    \renewcommand{\thefootnote}{*\arabic{footnote}}
    \renewcommand{\figurename}{Fig\ }
    \renewcommand{\tablename}{Tbl}
    \newcommand{\figref}[1]{Fig\ \ref{#1}}
    \newcommand{\tabref}[1]{Tbl\ \ref{#1}}

\makeatletter
    \renewcommand{\thefigure}{%
    \thesection.\arabic{figure}}
    \@addtoreset{figure}{section}

    \renewcommand{\thetable}{%
    \thesection.\arabic{table}}
    \@addtoreset{table}{section}

    \@addtoreset{lstlisting}{section}
\makeatother

\begin{document}
\begin{center}
    {\LARGE 学習指導案}
\end{center}
\begin{flushright}
    学校名:XX高等学校\\
    指導教員:溝口洸熙
\end{flushright}
\begin{description}
    \item[日時] \today 1限目 (8:50 - 9:50)
    \item[学級] 3年A組 (男子18名 女子22名)
    \item[生徒の実態] \ \\
        現在の状況をつらつらと.とりあえず,まぁ,なんとか,授業やってますてきな.
    \item[単元] \(n\)時間\\
        \begin{minipage}[t]{0.45\linewidth}
            \centering
            \begin{enumerate}
                \item オリエンテーション
                \item 集合理論写像とか
                \item 数理論理
                \item モデル
                \item うんち
            \end{enumerate}
        \end{minipage}
        \begin{minipage}[t]{0.45\linewidth}
            \centering
            \begin{enumerate}
                \setcounter{enumi}{5}
                \item 形式体系におけるしょーめー
                \item トートロジーと証明可能性
                \item グラフ理論
                \item 最大流問題とか
                \item 色々
            \end{enumerate}
        \end{minipage}
        \vspace{1em}
        \item[本時の目標]\
        \begin{enumerate}
            \item 色々できる
            \item 証明できる
            \item 足し算できる
        \end{enumerate}
        \item[評価基準]\
        \begin{itemize}
            \item 地震の
                  \begin{itemize}
                      \item 臨時情報
                      \item 対応
                            \begin{itemize}
                                \item できるんか?
                            \end{itemize}
                  \end{itemize}
            \item \(g(t)\)
            \item \(h(t)\)
        \end{itemize}
\end{description}
\newpage
\fontsize{10pt}{0cm}\selectfont

\noindent\textbf{指導手順}
\begin{table}[h]
    \fontsize{8pt}{0}\selectfont
    \newcommand{\activee}{0.2\textwidth}
    \newcommand{\valuee}{0.5\textwidth}
    \newcommand{\pointt}{0.23\textwidth}
    \centering
    \renewcommand{\arraystretch}{1.2}
    \begin{tabular}{|c|c|c|}
        \hline
        活動                                                                 & {指導内容} & {指導上の留意点及び評価} \\
        \hline
        \begin{minipage}{\activee}
            \textbf{導入}\\
            前時の学習の確認
        \end{minipage}                                           &
        \begin{minipage}{\valuee}
            前回の復習\\
            二次関数\(y=2x^2-2\)のグラフを描き,頂点の座標と軸の方程式を求めよ.
        \end{minipage} &
        \begin{minipage}{\pointt}
            \vspace{0.5em}
            \begin{itemize}
                \item 前時の評価を基に,不十分な生徒に机間指導の際,個別指導を行う.
                \item 前時の学習内容を確認しながら,答え合わせをする.
            \end{itemize}
            \vspace{0.5em}
        \end{minipage}                                                                                     \\
        \hline
        \begin{minipage}{\activee}
            \vspace{0.5em}
            \textbf{展開}\\
            グラフから関数の式\[y=a(x-p)^2\]を推測する.
        \end{minipage}                         &
        \begin{minipage}{\valuee}
            \vspace{0.5em}
            \begin{framed}
                \textbf{課題1}\hspace{1em}
                二次関数\(y=2x^2\)のグラフを\(x\)軸方向に1だけ平行移動したグラフを描く.
            \end{framed}
            \vspace{0.5em}
            \begin{framed}
                \textbf{課題2}\hspace{1em}
                これまでに学習したことを用いて,平行移動した関数の式を調べ,それはどのような式になるかを考える.
            \end{framed}
        \end{minipage}                                           &
        \begin{minipage}{\pointt}
            \vspace{0.5em}
            \begin{itemize}
                \item グラフを書くことで,平行移動の概念を理解させる.
                \item 3点\((0,0), (1,2), (2,8)\)がそれぞれどの点に移動するか考える.
            \end{itemize}
            \begin{framed}
                \textbf{関心・意欲・態度}\\
                \(x\)軸方向へ平行移動する二次関数のグラフについて関心を持ち,調べようとする.
            \end{framed}
        \end{minipage}                                                                                    \\
                                                                             &
        \begin{minipage}{\valuee}
            \textbf{予想される生徒の回答}\\
            \begin{itemize}
                \item 点の平行移動を調べる
                \item 頂点や軸の平行移動を調べる
                \item 一次関数を調べる
            \end{itemize}
            \vspace{0.5em}
        \end{minipage}                                      &
        \begin{minipage}{\pointt}
            いろいろ考えよう
        \end{minipage}                                                                                     \\
        \hline
        \begin{minipage}{\activee}
            \textbf{振り返り}\\
            二次関数の式とグラフの平行移動について理解する.
        \end{minipage}                     &
        \begin{minipage}{\valuee}
            \vspace{0.5em}
            \begin{framed}
                \textbf{発問}\hspace{1em}二次関数\(y=2(x-p)^2\)のグラフは
                \(y=2x^2\)のグラフを\(x\)軸方向にどのように平行移動したグラフとなるか
            \end{framed}
            \vspace{0.5em}
        \end{minipage}                                           &
        \begin{minipage}{\pointt}
            やる気ありますかい?
        \end{minipage}                                                                                     \\
        \hline
    \end{tabular}
\end{table}

\end{document}