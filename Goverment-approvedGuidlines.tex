\documentclass[a4j,14pt]{jsarticle}
%jbbok:書籍・jsarticle:論文,短い文書・jreport:レポート・letter:手紙・twocolumn:2段組
\usepackage{amsmath,amsfonts,amssymb,mathtools,ascmac,bm,float,comment,url,fancybox,calc,subcaption,multicol,physics,tablefootnote,caption,graphics,multirow,plext,multirow,array}
\usepackage{wrapfig}
\usepackage[T1]{fontenc}
\usepackage[top=15truemm,bottom=20truemm,left=15truemm,right=15truemm]{geometry}
\usepackage[dvipdfmx]{graphicx,color,hyperref}
\usepackage{tikz,listings,jlisting}
\usepackage{pxjahyper}
\usepackage{xcolor}
\usepackage{framed}
\hypersetup{
    colorlinks=true,
    citecolor=black,
    linkcolor=black,
    urlcolor=blue
}

    \usetikzlibrary{arrows}
    \usetikzlibrary{intersections,calc,arrows.meta,backgrounds,shapes.geometric,shapes.misc,positioning,fit,graphs}

    \setlength{\columnsep}{5mm}
    \columnseprule=0.1mm
\newcounter{countitem}
\setcounter{countitem}{0}
\newcommand{\countitem}[1][]{\refstepcounter{countitem}\Roman{countitem}}
\begin{document}
\begin{center}
    {\LARGE 学習指導案}
\end{center}
\begin{flushright}
    学校名:XX高等学校\\
    指導教員:溝口洸熙
\end{flushright}
\begin{description}
    \item[日時] \today 1限目 (8:50 - 9:50)
    \item[学級] 3年A組 (男子18名 女子22名)
    \item[生徒の実態] \ \\
        現在の状況をつらつらと.とりあえず,まぁ,なんとか,授業やってますてきな.
    \item[単元] \(n\)時間\\
        \begin{minipage}[t]{0.45\linewidth}
            \centering
            \begin{enumerate}
                \item オリエンテーション
                \item 集合理論とか写像とか
                \item 数理論理
                \item モデル
                \item うんち
            \end{enumerate}
        \end{minipage}
        \begin{minipage}[t]{0.45\linewidth}
            \centering
            \begin{enumerate}
                \setcounter{enumi}{5}
                \item 形式体系におけるしょーめー
                \item トートロジーと証明可能性
                \item グラフ理論
                \item 最大流問題とか
                \item 色々
            \end{enumerate}
        \end{minipage}
        \vspace{1em}
        \item[本時の目標]\
        \begin{enumerate}
            \item 色々できる
            \item 証明できる
            \item 足し算できる
        \end{enumerate}
        \item[評価基準]\
        \begin{itemize}
            \item 地震の
                  \begin{itemize}
                      \item 臨時情報
                      \item 対応
                            \begin{itemize}
                                \item できるんか?
                            \end{itemize}
                  \end{itemize}
            \item \(g(t)\)
            \item \(h(t)\)
        \end{itemize}
\end{description}
\newpage
\fontsize{10pt}{0cm}\selectfont

\noindent\textbf{指導手順}
\begin{table}[h]
    \fontsize{8pt}{0}\selectfont
    \newcommand{\activee}{0.1\textwidth}
    \newcommand{\valuee}{0.41\textwidth}
    \newcommand{\pointt}{0.24\textwidth}
    \newcommand{\evaluationn}{0.1\textwidth}
    \centering
    \begin{tabular}{|c|c|c|c|}
        \hline
        活動                       & {指導内容} & {指導場の留意点} & 評価の観点 \\
        \hline
        \begin{minipage}{\activee}
            \centering
            \pbox<t>{導入}
        \end{minipage} &
        \begin{minipage}{\valuee}
            前回の復習
        \end{minipage}  &
        \begin{minipage}{\pointt}
            この子はいいねぇ
        \end{minipage}  &
        \begin{minipage}{\evaluationn}
            関心・意欲
        \end{minipage}                                           \\
        \hline
    \end{tabular}
\end{table}
\end{document}