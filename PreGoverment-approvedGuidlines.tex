\RequirePackage{luatex85}
\documentclass[paper=a4,fontsize=10.5pt]{jlreq}
\usepackage{amsmath,amsfonts,amssymb,mathtools,ascmac,bm,fancybox,calc,multicol,physics,array}
\usepackage[top=20truemm,bottom=20truemm,left=15truemm,right=15truemm]{geometry}
\usepackage{graphicx,color}
\usepackage{TeachersGuide}
\usepackage{tikz,listing,wrapfig,float,xcolor}
\usepackage{url,subcaption,multirow,framed}
\usepackage[unicode,hidelinks,pdfusetitle]{hyperref}
\usepackage{luatexja-fontspec,lltjext,longtable}
\hypersetup{
    colorlinks=true,
    citecolor=black,
    linkcolor=black,
    urlcolor=blue
}

    \usetikzlibrary{intersections,calc,arrows.meta,backgrounds,shapes.geometric,shapes.misc,positioning,fit,graphs,arrows}
    \setlength{\columnsep}{5mm}

    \columnseprule=0.1mm
    \ltjsetparameter{jacharrange={-2}} %日本語以外を欧文扱い

    \renewcommand{\thefootnote}{*\arabic{footnote}}
    \renewcommand{\figurename}{Fig\ }
    \renewcommand{\tablename}{Tbl}
    \newcommand{\figref}[1]{Fig\ \ref{#1}}
    \newcommand{\tabref}[1]{Tbl\ \ref{#1}}

\makeatletter
    \renewcommand{\thefigure}{%
    \thesection.\arabic{figure}}
    \@addtoreset{figure}{section}

    \renewcommand{\thetable}{%
    \thesection.\arabic{table}}
    \@addtoreset{table}{section}

    \@addtoreset{lstlisting}{section}
\makeatother

\begin{document}
\showTitle{溝口洸熙}{数学}
\begin{showGoal}
    いろいろわかるようになる.
\end{showGoal}
\begin{TeachingProcedures}
    \begin{activitycol}
        \textbf{導入}\\
        前時の学習の確認
    \end{activitycol}&
    \begin{contentcol}
        \textbf{前回の復習}\\
        二次関数\(f(x)=2x^2-2\)のグラフを描き,頂点の座標と軸の方程式をを求めよ.
    \end{contentcol}&
    \begin{pointcol}
        \begin{itemize}
            \item 前時の評価に基づき,不十分な生徒に.....
            \item 前時の学習内容を確認しながら,答え合わせ.
        \end{itemize}
    \end{pointcol}\\
    \begin{activitycol}
        復習と,引っかかり
    \end{activitycol}&
    \begin{contentcol}
        {\begin{equation}
                \begin{aligned}
                    f(x) & = g(x)\circ h(x) \\
                    g(x) & = \sin x         \\
                    h(x) & = \cos x
                \end{aligned}
            \end{equation}
        }
    \end{contentcol}&
    \begin{pointcol}
        \begin{framed}
            課題1\\
            \[f(x)=(x-p)^2\]
        \end{framed}
    \end{pointcol}\\
    \hline
    \begin{activitycol}
        \textbf{展開}\\
        いろいろ展開しまする.
    \end{activitycol}&
    \begin{contentcol}
        \begin{equation}
            \begin{aligned}
                  & 4\pi k Q              \\
                F & =ma                   \\
                F & =G\dfrac{m_1m_2}{r^2}
            \end{aligned}
        \end{equation}
    \end{contentcol}&
    \begin{pointcol}
        \begin{framed}
            課題2\\
            \begin{equation*}
                f(x)=\left(\begin{aligned}
                    0            & x=0     \\\\
                    \dfrac{1}{x} & x\neq 0
                \end{aligned}
                \right.
            \end{equation*}
        \end{framed}
    \end{pointcol}\\
    \begin{activitycol}
        \textbf{展開}\\
        いろいろ展開しまする.
    \end{activitycol}&
    \begin{contentcol}
        \begin{equation}
            \begin{aligned}
                  & 4\pi k Q              \\
                F & =ma                   \\
                F & =G\dfrac{m_1m_2}{r^2}
            \end{aligned}
        \end{equation}
    \end{contentcol}&
    \begin{pointcol}
        \begin{framed}
            課題2\\
            \begin{equation*}
                f(x)=\begin{cases}
                    0            & x=0     \\\\
                    \dfrac{1}{x} & x\neq 0
                \end{cases}
            \end{equation*}
        \end{framed}
    \end{pointcol}\\
    \hline
    \begin{activitycol}
        \textbf{展開}\\
        いろいろ展開しまする.
    \end{activitycol}&
    \begin{contentcol}
        \begin{equation}
            \begin{aligned}
                  & 4\pi k Q              \\
                F & =ma                   \\
                F & =G\dfrac{m_1m_2}{r^2}
            \end{aligned}
        \end{equation}
    \end{contentcol}&
    \begin{pointcol}
        \begin{framed}
            課題2\\
            \begin{equation*}
                f(x)=\begin{cases}
                    0            & x=0     \\\\
                    \dfrac{1}{x} & x\neq 0
                \end{cases}
            \end{equation*}
        \end{framed}
    \end{pointcol}\\
    \begin{activitycol}
        \textbf{展開}\\
        いろいろ展開しまする.
    \end{activitycol}&
    \begin{contentcol}
        \begin{equation}
            \begin{aligned}
                  & 4\pi k Q              \\
                F & =ma                   \\
                F & =G\dfrac{m_1m_2}{r^2}
            \end{aligned}
        \end{equation}
    \end{contentcol}&
    \begin{pointcol}
        \begin{framed}
            課題2\\
            \begin{equation*}
                f(x)=\begin{cases}
                    0            & x=0     \\\\
                    \dfrac{1}{x} & x\neq 0
                \end{cases}
            \end{equation*}
        \end{framed}
    \end{pointcol}\\
    \begin{activitycol}
        \textbf{展開}\\
        いろいろ展開しまする.
    \end{activitycol}&
    \begin{contentcol}
        \begin{equation}
            \begin{aligned}
                  & 4\pi k Q              \\
                F & =ma                   \\
                F & =G\dfrac{m_1m_2}{r^2}
            \end{aligned}
        \end{equation}
    \end{contentcol}&
    \begin{pointcol}
        \begin{framed}
            課題2\\
            \begin{equation*}
                f(x)=\begin{cases}
                    0            & x=0     \\\\
                    \dfrac{1}{x} & x\neq 0
                \end{cases}
            \end{equation*}
        \end{framed}
    \end{pointcol}\\
    \hline
\end{TeachingProcedures}
\end{document}